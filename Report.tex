\documentclass[10pt, twocolumn, twoside]{article}
%List of article class defaults (with some useful options in brackets) = A4, 10pt [11pt, 12pt], onecolumn [twocolumn], oneside [twoside], notitlepage, portrait, equations-centre-rightlabel, final [draft]. 
\usepackage{amsmath}
\usepackage[left=0.70cm, right=0.70cm, top=0.70cm, bottom=0.70cm]{geometry} % sort out the too large LaTeX default margins.
\usepackage{graphicx} %for importing graphics, see \begin{figure} below.
\usepackage{amssymb,amsmath} % add some standard packages for maths and symbols
\usepackage[colorlinks=true,linkcolor=blue]{hyperref} % turns all latex references into hyperlinks, useful if you have e.g. a large document with a large table of contents. Note that a known Apple bug is that the hyperlinks exist in the pdf, but don't they show up under the Apple pdf reader, Preview. Try Adobe Reader for your pdf on a Mac.

\usepackage{lipsum} % for generating dummy text, you can delete all lipsum commands in your final document, including this one.

\author{E. Davies\\PH1999 report}

\title{Dark Matter Project}

\date{\today}

%everything before this line is called the 'preamble'. Now we begin the document. 

\begin{document}

\maketitle %this uses the author, title, date information given in the preamble to make a title.

\begin{abstract}
TBF
\end{abstract}

%The basic structure of LaTeX is that of \beginning an {environment}, and ending it. Above is the 'abstract' environment. Note the figure, table and equation environments below. 



\begin{itemize}
  \item M=first mass(kg)
  \item m=second mass(kg)
  \item v=velocity(m/s)
  \item r=radius(m)
  \item F=Force(N)
  \item a=acceleration(m/s^2)
  
  \item G=Gravitational constant(6.67x10^-11)
\end{itemize}


\section{Weak 1}
The relationship between acceleration, velocity and radius in uniform circular motion is:
\label{GaussE}
\begin{equation}
a=v^2/r
\end{equation}
The gravitational force between two point masses is:
%an example displayed equation
\begin{equation}
\label{GaussE}
F=GMm/r^2
\end{equation}
from these you can calculate the equation for velocity as a function of mass and radius:

\begin{equation*}
\label{GaussE}
\begin{aligned}
F=GMm/r^2 ----F=ma ----a=v^2/r
\end{aligned}
\end{equation*}
\begin{equation*}
\label{GaussE}
\begin{aligned}
F=mv^2/r
\end{aligned}
\end{equation*}
\begin{equation*}
\label{GaussE}
\begin{aligned}
GMm/r^2=mv^2/r
\end{aligned}
\end{equation*}
\begin{equation*}
\label{GaussE}
\begin{aligned}
GM/r^2=v^2/r
\end{aligned}
\end{equation*}
\begin{equation*}
\label{GaussE}
\begin{aligned}
GM/r=v^2
\end{aligned}
\end{equation*}
\begin{equation*}
\label{GaussE}
\begin{aligned}
\sqrt(GM/r)=v
\end{aligned}
\end{equation*}
Given this you can calculate the mass within a sphere, when talking about galaxy's,
this is visualised as a gas cloud with density ρ and the mass is the all the "particles"
within the radius this is called Mc(r) for contained mass within the radius.
v=Volume(kg/m^3) V=Velocity(m/s):

\begin{equation*}
\label{GaussE}
\begin{aligned}
\rho= m/v----v=4/3\pi r^3---\sqrt(GM/r)=V
\end{aligned}
\end{equation*}
\begin{equation*}
\label{GaussE}
\begin{aligned}
 m=\rho v
\end{aligned}
\end{equation*}
\begin{equation*}
\label{GaussE}
\begin{aligned}
 m= 4/3\pi r^3\rho
\end{aligned}
\end{equation*}
\begin{equation*}
\label{GaussE}
\begin{aligned}
\sqrt(4/3\pi r^3\rho G/r)=V
\end{aligned}
\end{equation*}
\begin{equation*}
\label{GaussE}
\begin{aligned}
\sqrt(4/3\pi r^2\rho G)=V
\end{aligned}
\end{equation*}
\begin{equation*}
\label{GaussE}
\begin{aligned}
r\sqrt(4/3\pi \rho G)=V
\end{aligned}
\end{equation*}
the velocity is now positively directly proportion to the radius
this means you can predict a velocity based on the radius of a galaxy.





\section{Experimental setup}
Describe the key features of the equipment you used. 

As shown here, an empty line in the source code starts a new paragraph with a first line indent in the output. The primary source of information about LateX is 'ctan.org' , the Comprehensive TeX Archive Network, it contains detailed information on every aspect of LaTeX, but google will usually provide LaTeX novices with simple examples elsewhere for most of what you want to do. The American Mathematical Society short guide to maths typesetting can be found at \href{<ftp://ftp.ams.org/pub/tex/doc/amsmath/short-math-guide.pdf>}{here}. It will give you almost every symbol you need. Good guidance for novices can be found at \href{<https://en.m.wikibooks.org/wiki/LaTeX/Basics>}{here}, which includes installation advice, or the book by Lesley Lamport is the classic reference. Though hard work at first, LaTex ultimately makes larger scientific documents (such as your final year project reports) far easier to write and with a far classier presentation.

The following dummy text is just to check I am getting the margins right. You can delete all lipsum commands in your final document.


\vspace{1cm} % a vertical space to separate the dummy text for clarity.


\lipsum[1-3]



%here I import a figure, it is easiest to put the graphic file in the same directory.
%\begin{figure*}[ht] %asterisk implies width is two columns
\begin{figure}[ht] %h= put it approx here, t=top of page, b=bottom of page, ! = "force it".
\includegraphics[width=\columnwidth]{myFigure}
\caption[width=\columnwidth]{Figure spanning one column. How I feel when my python data analysis finally works! The caption text should be sufficiently complete that it makes sense without the main body text. Typically, data would imported into python, manipulated, and plotted with matplotlib, along with best fit lines. Use python to produce an image file (typically a jpg, png or eps) which can then be imported into LaTeX.}
%\end{figure*}
\end{figure}




\section{Measurement Methods}
Describe what you aimed to measure and how you did it. Here is a footnote \footnote{I don't understand what I did, but it worked!}.

\lipsum[1-2]
\section{Results and Conclusions}
Provide plots of data and best fit lines, describe the results, draw conclusions from the data.



\begin{table}[!ht] %here is a simple table, useful for listing experimental and best fit parameters, for example.
  \begin{center}
    \begin{tabular}{| l c r c |} %justify the data in the columns:  l,c,r = left, centre, right, (inspect the output table carefully to see this), | = vertical line (the sides of the table) Add gridlines with | and \hline
    \hline
    V /Volts & I /Amperes & R /Ohms & Element \\
    1 & 2 & 3 & H \\
    4 & 5 & 6 & He \\
    7 & 8 & 9 & Li \\
    10 & 11 & 12 & Na \\
    \hline
    \end{tabular}
  \end{center}
  \caption{A simple table}
\end{table}



\lipsum[1-4]

\begin{thebibliography}{1}

\bibitem{maxwell} James Clerk Maxwell, \emph{A dynamical theory of the electromagnetic field},  Phil. Trans. Roy. Soc. {\bf 155}, 459-512, (1865).% add a \bibitem{labelxyz} for use with \cite{labelxyz}

\end{thebibliography}

\end{document}
